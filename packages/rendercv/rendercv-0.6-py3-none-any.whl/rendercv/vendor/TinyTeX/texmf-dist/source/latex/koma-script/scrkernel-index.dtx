% \iffalse meta-comment
% ======================================================================
% scrkernel-index.dtx
% Copyright (c) Markus Kohm, 2002-2023
%
% This file is part of the LaTeX2e KOMA-Script bundle.
%
% This work may be distributed and/or modified under the conditions of
% the LaTeX Project Public License, version 1.3c of the license.
% The latest version of this license is in
%   http://www.latex-project.org/lppl.txt
% and version 1.3c or later is part of all distributions of LaTeX 
% version 2005/12/01 or later and of this work.
%
% This work has the LPPL maintenance status "author-maintained".
%
% The Current Maintainer and author of this work is Markus Kohm.
%
% This work consists of all files listed in MANIFEST.md.
% ======================================================================
%%% From File: $Id: scrkernel-index.dtx 4052 2023-05-05 11:41:51Z kohm $
%<option>%%%            (run: option)
%<body>%%%            (run: body)
%<*dtx>
\ifx\ProvidesFile\undefined\def\ProvidesFile#1[#2]{}\fi
\begingroup
  \def\filedate$#1: #2-#3-#4 #5${\gdef\filedate{#2/#3/#4}}
  \filedate$Date: 2023-05-05 13:41:51 +0200 (Fr, 05. Mai 2023) $
  \def\filerevision$#1: #2 ${\gdef\filerevision{r#2}}
  \filerevision$Revision: 4052 $
  \edef\reserved@a{%
    \noexpand\endgroup
    \noexpand\ProvidesFile{scrkernel-index.dtx}%
                          [\filedate\space\filerevision\space
                           KOMA-Script source
                           (index)]%
  }%
\reserved@a
\documentclass[USenglish]{koma-script-source-doc}
\usepackage{babel}
\setcounter{StandardModuleDepth}{2}
\begin{document}
\DocInput{scrkernel-index.dtx}
\end{document}
%</dtx>
% \fi
%
% \changes{v2.95}{2002/07/01}{new by splitting \file{scrclass.dtx}}
% \changes{v3.36}{2022/02/15}{switch over from \cls*{scrdoc} to
%   \cls*{koma-script-source-doc}}
% \changes{v3.36}{2022/02/15}{whole implementation documentation in English}
% \changes{v3.40}{2023/04/17}{guide names changed}
%
% \GetFileInfo{scrkernel-index.dtx}
% \title{Index Code of the 
%   \href{https://komascript.de}{\KOMAScript} Classes}
% \author{\href{mailto:komascript@gmx.info}{Markus Kohm}}
% \date{Revision \fileversion{} of \filedate}
% \maketitle
% \begin{abstract}
%   The index code of the \KOMAScript{} classes in \file{scrkernel-index.dtx}
%   is somehow different from the code of the standard classes. It provides
%   several options and other configuration possibilities.
% \end{abstract}
% \tableofcontents
%
% \section{User Manual}
%
% You can find the user documentation the commands implemented here in the
% \KOMAScript{} manual, either the German \file{scrguide-de.pdf} or the
% English \file{scrguide-en.pdf}.
% 
% \MaybeStop{\PrintIndex}
%
% \section{Implementation of the Index}
%
% The letter class \cls*{scrlttr2} does not provide an index. All other
% classes do.
%
%    \begin{macrocode}
%<*!letter>
%    \end{macrocode}
%
%
% \subsection{Options}
%
% \begin{option}{index}
% \changes{v2.98c}{2008/03/04}{added}
% \changes{v3.17}{2015/03/12}{internal value storage}
% \changes{v3.18}{2015/06/15}{new value \opt{numbered}}
% \changes{v3.18}{2015/06/15}{new value \opt{leveldown}}
% \changes{v3.18}{2015/06/15}{new value \opt{standardlevel}}
% \changes{v2.7a}{2001/01/04}{improved link generation with \cls*{scrbook} or
%   \cls*{scrreprt} and \pkg{hyperref}}
% \changes{v2.98c}{2008/03/04}{redefining \cs{idx@@heading} instead of
%   \cs{idx@@heading}}
% \changes{v3.18}{2015/06/15}{new value \opt{leveldown}}
% \changes{v3.39}{2022/11/11}{initial dot removed from member argument of
%   option storage commands}
% The option allow several values for several features:
% \begin{labeling}{\opt{=standardlevel}:}
% \item[\opt{=default}:]       use not numbered heading without ToC entry, but
%   without affecting the heading level;
% \item[\opt{=totoc}:]         use not numbered heading with ToC entry, but
%   without affecting the heading level;
% \item[\opt{=numbered}:]      use numbered heading (with ToC entry), but
%   without affecting the heading level;
% \item[\opt{=leveldown}:]     use a subheading (e.g. \cs{subsection} instead
%   of \cs{section} in \cls*{scrartcl}), without affecting numbered or not and
%   ToC entry or not;
% \item[\opt{=standardlevel}:] don't use a subheading, without affecting
%   numbered or not and ToC entry or not.
% \end{labeling}
% To do so, the option redefines \cs{idx@heading}.
% \begin{macro}{\idx@@heading}
% \changes{v2.98c}{2008/03/04}{added}
% \changes{v3.18}{2015/06/15}{using \opt{leveldown}}
% This is the macro to print the heading of the index (nothing else). It has
% one mandatory argument: the text to be used as heading. There will be one
% additional code level above this: \cs{idx@heading}, that will be defined in
% the \texttt{body} run. \cs{idx@heading} has to be defined before, so it can
% be redefined by the \opt{index} option. In opposite to the standard classes
% we want to use \cs{chapter} in two column mode. This can be done, because of
% significant changes of \cs{chapter} definition. Note: If someone uses the
% standard classes hack of package \pkg*{scrhack} or even another headings
% package like \pkg*{titlesec} this could fail. But this is not my problem,
% because usage of such packages and hacks is not recommended at all.
%    \begin{macrocode}
%<*option>
\newcommand*{\idx@@heading}{%
  \ifidx@leveldown
%<article>    \expandafter\subsection
%<book|report>    \expandafter\section
  \else
%<article>    \expandafter\section
%<book|report>    \expandafter\chapter
  \fi*%
}
%    \end{macrocode}
% \begin{macro}{\idx@@mark,\idx@@forcemark,\idx@@forcemarknumbered}
% \changes{v3.41}{2023/05/05}{new}
% We also need a command to do the mark for the running head to be used if a
% sectioning command without automatic usage of \cs{chaptermark},
% \cs{sectionmark} or \cs{subsectionmark} is used. For headings with automatic
% usage of \cs{chaptermark}, \cs{sectionmark} or \cs{subsectionmark} it should
% not be used, despite the fact, that this usually would result in not setting
% both marks. The command has the same argument as \cs{idx@@heading}. This is
% the same code, previously used in \cs{idx@heading}.
%    \begin{macrocode}
\newcommand*{\idx@@forcemark}[1]{%
  \ifidx@leveldown
%<report|book>    \addsecmark{#1}%
%<*article>
    \begingroup
      \value{secnumdepth}=-\maxdimen
      \subsectionmark{#1}%
    \endgroup
%</article>
  \else
    \@mkdouble{\MakeMarkcase{#1}}%
  \fi
}
\newcommand*{\idx@@forcemarknumbered}[1]{%
  \ifidx@leveldown
%<report|book>    \sectionmark{#1}%
%<article>    \subsectionmark{#1}%
  \else
%<report|book>    \chaptermark{#1}%
%<article>    \sectionmark{#1}%
  \fi
}
\newcommand*{\idx@@mark}{}
\let\idx@@mark\idx@@forcemark
%    \end{macrocode}
% \end{macro}
% \begin{macro}{\idx@leveldowntrue,\idx@leveldownfalse,\ifidx@leveldown}
% \changes{v3.18}{2015/06/15}{added}
% The boolean switch \cs{ididx@leveldown} is needed to distinguish the heading
% level to be used (option values \opt{leveldown} and \opt{standardlevel}).
%    \begin{macrocode}
\newif\ifidx@leveldown
\KOMA@key{index}{%
  \KOMA@set@ncmdkey{index}{@tempa}{%
    {notoc}{0},{nottotoc}{0},{default}{0},{plainheading}{0},%
    {totoc}{1},{toc}{1},{notnumbered}{1},%
    {numbered}{2},{totocnumbered}{2},{numberedtotoc}{2},{tocnumbered}{2},%
    {numberedtoc}{2},%
    {leveldown}{3},%
    {standardlevel}{4}%
  }{#1}%
  \ifx\FamilyKeyState\FamilyKeyStateProcessed
    \ifcase \@tempa\relax
      \KOMA@kav@remove{\KOMAClassFileName}{toc}{noindex}%
      \KOMA@kav@remove{\KOMAClassFileName}{toc}{indexnumbered}%
      \KOMA@kav@remove{\KOMAClassFileName}{toc}{index}%
      \KOMA@kav@remove{\KOMAClassFileName}{index}{notoc}%
      \KOMA@kav@remove{\KOMAClassFileName}{index}{numbered}%
      \KOMA@kav@remove{\KOMAClassFileName}{index}{toc}%
      \KOMA@kav@add{\KOMAClassFileName}{toc}{noindex}%
      \KOMA@kav@add{\KOMAClassFileName}{index}{notoc}%
      \renewcommand*{\idx@@heading}{%
        \ifidx@leveldown
%<article>        \expandafter\subsection
%<book|report>        \expandafter\section
        \else
%<article>        \expandafter\section
%<book|report>        \expandafter\chapter
        \fi*%
      }%
      \let\idx@@mark\idx@@forcemark
    \or
      \KOMA@kav@remove{\KOMAClassFileName}{toc}{noindex}%
      \KOMA@kav@remove{\KOMAClassFileName}{toc}{indexnumbered}%
      \KOMA@kav@remove{\KOMAClassFileName}{toc}{index}%
      \KOMA@kav@remove{\KOMAClassFileName}{index}{notoc}%
      \KOMA@kav@remove{\KOMAClassFileName}{index}{numbered}%
      \KOMA@kav@remove{\KOMAClassFileName}{index}{toc}%
      \KOMA@kav@add{\KOMAClassFileName}{toc}{index}%
      \KOMA@kav@add{\KOMAClassFileName}{index}{toc}%
      \renewcommand*{\idx@@heading}{%
        \ifidx@leveldown
%<article>        \expandafter\expandafter\expandafter\subsection\expandafter*%
%<book|report>        \expandafter\addsec
        \else
%<article>        \expandafter\addsec
%<book|report>        \expandafter\addchap
        \fi
      }%
      \let\idx@@mark\idx@@forcemark
    \or
      \KOMA@kav@remove{\KOMAClassFileName}{toc}{noindex}%
      \KOMA@kav@remove{\KOMAClassFileName}{toc}{indexnumbered}%
      \KOMA@kav@remove{\KOMAClassFileName}{toc}{index}%
      \KOMA@kav@remove{\KOMAClassFileName}{index}{notoc}%
      \KOMA@kav@remove{\KOMAClassFileName}{index}{numbered}%
      \KOMA@kav@remove{\KOMAClassFileName}{index}{toc}%
      \KOMA@kav@add{\KOMAClassFileName}{toc}{indexnumbered}%
      \KOMA@kav@add{\KOMAClassFileName}{index}{numbered}%
      \renewcommand*{\idx@@heading}{%
        \ifidx@leveldown
%<article>        \expandafter\subsection
%<book|report>        \expandafter\section
        \else
%<article>        \expandafter\section
%<book|report>        \expandafter\chapter
        \fi
      }%
      \let\idx@@mark\idx@@forcemarknumbered
    \or
      \KOMA@kav@remove{\KOMAClassFileName}{index}{leveldown}%
      \KOMA@kav@remove{\KOMAClassFileName}{index}{standardlevel}%
      \KOMA@kav@add{\KOMAClassFileName}{index}{leveldown}%
      \idx@leveldowntrue
    \or
      \KOMA@kav@remove{\KOMAClassFileName}{index}{leveldown}%
      \KOMA@kav@remove{\KOMAClassFileName}{index}{standardlevel}%
      \KOMA@kav@add{\KOMAClassFileName}{index}{standardlevel}%
      \idx@leveldownfalse
    \fi
  \fi
}
\KOMA@kav@add{\KOMAClassFileName}{toc}{noindex}
\KOMA@kav@add{\KOMAClassFileName}{index}{notoc}
\KOMA@kav@add{\KOMAClassFileName}{index}{standardlevel}
%    \end{macrocode}
% \end{macro}^^A \ifidx@leveldown …
% \end{macro}^^A \idx@@heading
% \end{option}^^A index
%
% \begin{option}{idxtotoc}
% \changes{v2.3h}{1995/01/21}{new}
% \changes{v2.98c}{2008/03/04}{obsolete}
% \changes{v3.01a}{2008/11/20}{deprecated}
% \changes{v3.99}{2022/11/16}{not with \KOMAScript~4}
%    \begin{macrocode}
%<!v4>\KOMA@DeclareDeprecatedOption{idxtotoc}{index=totoc}
%</option>
%    \end{macrocode}
% \end{option}
%
%
% \subsection{User commands and macros}
%
% \begin{command}{\setindexpreamble}
% \changes{v2.6c}{2000/06/10}{added}
% \changes{v2.6d}{2000/07/20}{\cs{let} replaced by \cs{newcommand*}}
% \changes{v2.95}{2002/07/01}{\cs{let} and \cs{def}}
% \begin{macro}{\index@preamble}
% \changes{v2.6c}{2000/06/10}{added}
% The index preamble is placed between the heading an the index itself. The
% user can define it using \cs{setindexpreamble}. \cs{index@preamble} is the
% storage. Note: \cs{setindexpreamble} does read the argument itself. This
% makes it easier to patch the user level command. But because of this it has
% to long.
%    \begin{macrocode}
%<*body>
\newcommand*{\index@preamble}{}
\let\index@preamble=\relax
\newcommand{\setindexpreamble}[1]{\def\index@preamble{#1}}
%    \end{macrocode}
% \end{macro}
% \end{command}
%
% \begin{macro}{\idx@heading}
% \changes{v2.3h}{1995/01/21}{added}
% \changes{v2.4c}{1996/04/04}{handling of running heads fixed}
% \changes{v2.4g}{1996/11/04}{two-columned index with \cls*{scrartcl} fixed}
% \changes{v2.6c}{2000/06/10}{support for index preamble after heading}
% \changes{v2.7j}{2001/05/27}{output of preamble moved to \env{theindex}}
% \changes{v2.9l}{2003/01/19}{\cs{if@twoside} replaced by \cs{if@openright}}
% \changes{v2.98c}{2008/03/04}{using \cs{idx@@heading}}
% \changes{v3.10}{2011/08/31}{using \cs{MakeMarkcase}}
% \changes{v3.10a}{2012/03/08}{brackets from previous change fixed}
% \changes{v3.12}{2013/09/24}{never make an index entry to the other ToC but
%   the table of contents}
% \changes{v3.25}{2017/10/13}{using \cs{@mkright} and \cs{@mkdouble}}
% \changes{v3.26}{2018/09/03}{usage of \cs{addsecmark} or \cs{subsectionmark}
%   instead of \cs{@mkright}}
% \changes{v3.41}{2023/05/05}{using \cs{idx@@mark}}
% \changes{v3.41}{2023/05/05}{\cs{cleardoublepage} or \cs{clearpage} only, if
%   not \opt{leveldown}}
% Macro to make the index heading and preamble.
%    \begin{macrocode}
\newcommand*{\idx@heading}{%
%<report|book>  \ifidx@leveldown\else\if@openright\cleardoublepage\else\clearpage\fi\fi
  \twocolumn[%
%<report|book>    \@chaptertolistsfalse
                  \idx@@heading{\indexname}]%
  \idx@@mark{\indexname}%
}
%    \end{macrocode}
% \end{macro}
%
% \begin{environment}{theindex}
% \changes{v2.8d}{2001/07/05}{\cs{indexpagestyle} instead of \pstyle{plain}}
% \changes{v2.3h}{1995/01/21}{using \cs{idx@heading}}
% \changes{v2.7j}{2001/05/27}{output of preamble added}
% \changes{v3.19}{2015/08/25}{using \cs{indexpagestyle} only if not empty}
% This definition is similar to the definition of the standard classes. As
% addition it supports a preamble text, a variable heading, and resets the
% paragraph end line fill.
%    \begin{macrocode}
\newenvironment{theindex}{%
  \if@twocolumn
    \@restonecolfalse
  \else
    \@restonecoltrue
  \fi
  \columnseprule \z@
  \columnsep 35\p@
%<book|report>  \setchapterpreamble{\index@preamble}%
  \idx@heading
%<article>  \index@preamble\par\nobreak
  \ifx\indexpagestyle\@empty\else\thispagestyle{\indexpagestyle}\fi
  \parindent\z@
  \setlength{\parskip}{\z@ \@plus .3\p@}%
  \setlength{\parfillskip}{\z@ \@plus 1fil}%
  \let\item\@idxitem
}{%
  \if@restonecol\onecolumn\else\clearpage\fi
}
%    \end{macrocode}
% \begin{macro}{\@idxitem}
% A new index entry
% \begin{command}{\subitem,\subsubitem}
% \changes{v2.3g}{1996/01/14}{using \cs{@idxitem} saves memory}
% sub-entry, sub-sub-entry,
% \begin{command}{\indexspace}
% or (vertical) distance.
%    \begin{macrocode}
\newcommand*\@idxitem{\par\hangindent 40\p@}
%    \begin{macrocode}
\newcommand*\subitem{\@idxitem \hspace*{20\p@}}
\newcommand*\subsubitem{\@idxitem \hspace*{30\p@}}
\newcommand*\indexspace{%
  \par \vskip 10\p@ \@plus5\p@ \@minus3\p@\relax
}
%    \end{macrocode}
% \end{command}^^A \indexspace
% \end{command}^^A \subitem,\subsubitem
% \end{macro}^^A \@idxitem
% \end{environment}^^A theindex
%
%
% \subsection{Language dependent terms}
%
% \begin{command}{\indexname}
% \changes{v3.36}{2022/02/15}{explicit definition for English languages}
% The name of the index (used by \cs{idx@heading}).
%    \begin{macrocode}
\newcommand*\indexname{Index}
\providecaptionname{american,australian,british,canadian,english,newzealand,%
  UKenglish,ukenglish,USenglish,usenglish}\indexname{Index}
%</body>
%    \end{macrocode}
% \end{command}
%
%
%    \begin{macrocode}
%</!letter>
%    \end{macrocode}
%
% \Finale
% \PrintChanges
% 
\endinput
% Local Variables:
% mode: doctex
% ispell-local-dictionary: "en_US"
% eval: (flyspell-mode 1)
% TeX-master: t
% TeX-engine: luatex-dev
% eval: (setcar (or (cl-member "Index" (setq-local TeX-command-list (copy-alist TeX-command-list)) :key #'car :test #'string-equal) (setq-local TeX-command-list (cons nil TeX-command-list))) '("Index" "mkindex %s" TeX-run-index nil t :help "makeindex for dtx"))
% End:


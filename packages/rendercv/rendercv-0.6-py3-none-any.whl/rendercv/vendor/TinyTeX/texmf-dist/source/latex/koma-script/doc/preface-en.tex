% ======================================================================
% preface-en.tex
% Copyright (c) Markus Kohm, 2008-2022
%
% This file is part of the LaTeX2e KOMA-Script bundle.
%
% This work may be distributed and/or modified under the conditions of
% the LaTeX Project Public License, version 1.3c of the license.
% The latest version of this license is in
%   http://www.latex-project.org/lppl.txt
% and version 1.3c or later is part of all distributions of LaTeX
% version 2005/12/01 or later and of this work.
%
% This work has the LPPL maintenance status "author-maintained".
%
% The Current Maintainer and author of this work is Markus Kohm.
%
% This work consists of all files listed in MANIFEST.md.
% ======================================================================

\KOMAProvidesFile{preface-en.tex}
                 [$Date: 2022-06-05 12:40:11 +0200 (So, 05. Jun 2022) $
                  preface to dedicated version]
\translator{Markus Kohm\and Karl Hagen\and DeepL}

\addchap{Preface to \KOMAScript~3.36 and 3.37}

With \KOMAScript~3.36 a phase of major rebuilds of the sources of
\KOMAScript{} was initiated. It started with the sources of the classes and
packages. Not only a conversion to version~3 of the package \Package{doc} was
done. The original source code documentation class \Class{scrdoc} was also
made obsolete and replaced by \Class{koma-script-source-doc}. The complete
source code documentation was also transferred to English or recreated in
English. In the course of this change, some of the sources were also
reorganized within the files or moved to new files. In the process, many
dozens of new notes about undone tasks were added. Whether I will ever be able
to work through all of them myself remains to be seen.

Due to the massive modifications to the sources, it could not be ruled out
from the outset that new errors could creep in. Due to the acute shortage of
beta testers, this fear has unfortunately come true. Whether all these bugs
have been fixed in the meantime is hard to say.

More or less in parallel, a new test structure based on \Package{l3build} was
built. This should ensure in the future that once reported errors do not occur
again in the future.

With \KOMAScript~3.37 I started to restructure the sources of the manual. For
the generation of the German and English user manual including the complete
examples with PDF now also \Package{l3build} is used. In addition, a flat
hierarchy is used for the user guides in all languages. Thus there are no
different files with the same filename in the sources anymore. This is not
only to satisfy CTAN requirements. With this it is also possible for the first
time in a long time to generate the manuals from the CTAN sources of
\KOMAScript.

Due to the problems with the finiteness of a single developer's time already
explained in the preface to \KOMAScript~3.28, I will therefore continue to
concentrate on bug fixing, the necessary reorganisation of the sources and
compatibility with new \LaTeX{} kernel versions in the future. Especially with
the latter, I now have the support of Marei Peischl, who is already very busy
with her own projects. She also wrote the original code for the illustrations
of pseudo-lengths and variables, for which I thank her very much. This finally
made it possible for me to fulfil a long-cherished wish of many users. With a
few changes, the pseudo lengths in the illustration are now linked to the
corresponding explanations in the text.


By largely abstaining from new functions, the effort for documenting them
naturally also dwindles. Readers of this free, screen version, however, still
have to live with some restrictions. So some information\,---\,mainly intended
for advanced users or capable of turning an ordinary user into an advanced
one\,---\,is reserved for the printed book, which currently exists only in
German. As a result, some links in this manual lead to a page that simply
mentions this fact. In addition, the free version is scarcely suitable for
making a hard-copy. The focus, instead, is on using it on screen, in parallel
with the document you are working on. It still has no optimized wrapping but
is almost a first draft, in which both the paragraph and page breaks are in
some cases quite poor. Corresponding optimizations are reserved for the German
book editions.


The biggest thanks go to my family and above all to my wife. They absorb all
my unpleasant experiences on the Internet. They have also tolerated it for
more than 25~years, when I am again not approachable, because I am completely
lost in \KOMAScript{} or some \LaTeX{} problems. The fact that I can afford to
invest an incredible amount of time in such a project is entirely thanks to my
wife.

\bigskip\noindent
Markus Kohm, Neckarhausen in May 2022.

\endinput

%%% Local Variables: 
%%% mode: latex
%%% TeX-master: "scrguide-en.tex"
%%% coding: utf-8
%%% ispell-local-dictionary: "en_GB"
%%% eval: (flyspell-mode 1)
%%% End: 

% Copyright 2019 by Till Tantau
%
% This file may be distributed and/or modified
%
% 1. under the LaTeX Project Public License and/or
% 2. under the GNU Public License.
%
% See the file doc/generic/pgf/licenses/LICENSE for more details.

\ProvidesPackageRCS{pgfsys.code.tex}


% Load key mechanism
\ifdefined\pgfkeysloaded\else
  \input pgfkeys.code.tex
\fi

% "pgf" is a family
\pgfkeys{/pgf/.is family}
\def\pgfset{\pgfqkeys{/pgf}}

% This if is needed *very* early
\newif\ifpgfpicture


%
% The following conversion functions are used to convert from TeX
% dimensions to postscript/pdf points.
%
{\catcode`\p=12\catcode`\t=12\gdef\Pgf@geT#1pt{#1}}

\def\pgf@sys@tonumber#1{\expandafter\Pgf@geT\the#1}

\def\pgf@sys@bp@correct#1{#1=0.99627#1}

\def\pgf@sys@bp#1{%
  {%
    \pgf@x=#1\relax%
    \pgf@x=0.99627\pgf@x%
    \edef\temp{\expandafter\Pgf@geT\the\pgf@x\space}%
    \pgfutil@toks@\expandafter\expandafter\expandafter{\expandafter\pgfsysprotocol@currentprotocol\temp}%
    \xdef\pgfsysprotocol@currentprotocol{\the\pgfutil@toks@}%
  }%
}
\def\pgf@sys@pt#1{%
  {%
    \pgf@x=#1\relax%
    \edef\temp{\expandafter\Pgf@geT\the\pgf@x\space}%
    \pgfutil@toks@\expandafter\expandafter\expandafter{\expandafter\pgfsysprotocol@currentprotocol\temp}%
    \xdef\pgfsysprotocol@currentprotocol{\the\pgfutil@toks@}%
  }%
}


% internal dimensions that are always present when pgsys is loaded.
\newdimen\pgf@x
\newdimen\pgf@y
\newdimen\pgf@xa
\newdimen\pgf@ya
\newdimen\pgf@xb
\newdimen\pgf@yb
\newdimen\pgf@xc
\newdimen\pgf@yc
\newdimen\pgf@xd
\newdimen\pgf@yd

\newwrite\w@pgf@writea
\newread\r@pgf@reada
\let\pgfutil@inputcheck=\r@pgf@reada

% internal counters that are always present when pgfsys is loaded
\newcount\c@pgf@counta
\newcount\c@pgf@countb
\newcount\c@pgf@countc
\newcount\c@pgf@countd

\newtoks\t@pgf@toka
\newtoks\t@pgf@tokb
\newtoks\t@pgf@tokc

% Ensure that math registers are the same (math is broken in case it
% is loaded first)
\let\pgfmath@x\pgf@x
\let\pgfmath@xa\pgf@xa
\let\pgfmath@xb\pgf@xb
\let\pgfmath@xc\pgf@xc

\let\pgfmath@y\pgf@y
\let\pgfmath@ya\pgf@ya
\let\pgfmath@yb\pgf@yb
\let\pgfmath@yc\pgf@yc

\let\c@pgfmath@counta\c@pgf@counta
\let\c@pgfmath@countb\c@pgf@countb
\let\c@pgfmath@countc\c@pgf@countc
\let\c@pgfmath@countd\c@pgf@countd





% In the following dummy definitions of the pgf system commands are
% given. The individual drivers must overwrite these definitions,
% preferably all of them. If some system command is used that has not
% been overwritten, the command fails with a warning.
%
% In some cases, some pgf system commands are just shorthands for a
% series of more complicated commands. In this case, a default
% implementation is given in this file, but a driver may choose to
% overwrite this default implementation with a more efficient
% version. For example, a closestroke is the same as a closepath
% followed by a stroking, but in pdf there is a special shorthand for
% this. So, the pdf driver overwrites \pgfsys@closestroke, while the
% postscript driver does not.
%
% The commands for which default implementations are available are
% given at the end.

\def\pgf@sys@fail#1{%
  \expandafter\ifx\csname pgf@sys@fail@#1\endcsname\pgfutil@empty%
  \else%
    \expandafter\global\expandafter\let\csname pgf@sys@fail@#1\endcsname=\pgfutil@empty%
    \pgfwarning{Your graphic driver \pgfsysdriver\space does not support #1. This warning is given only once}%
  \fi%
}



%
% Invocation commands
%

\def\pgfsys@invoke{\pgf@sys@fail{invoking specials}}
% This command is used whenever some literal text needs to be inserted
% into the resulting .pdf, .ps, or .dvi file. Note that most
% \pgfsys@xxxx commands will not directly call this command, but,
% rather, use the protocolling mechanism to record their literal
% text. It is then left to the discretion of the protocolling
% mechanism to insert the literal text at some appropriate, later
% time.
%
% This command should only be called when \pgfsys@beginpicture has
% been called previously and when the picture has not been suspended
% using \pgfsys@beginhbox.

\def\pgfsys@outerinvoke{\pgfsys@invoke}
% This command can be used to insert protocolled graphic commands
% outside any picture. This is extremely dangerous, but sometimes
% useful. By default, this command does the same as inside pictures.



%
%
% Path construction commands
%
%

\def\pgfsys@moveto#1#2{\pgf@sys@fail{path constructions}}
% This command is used to start a path at a specific point (#1,#2)
% or to move the current point of the current path to (#1,#2)
% without drawing anything upon stroking (the current path is
% `interrupted').
%
% The arguments must be dimensions (like |10pt|).
%
% Example:
%
% \pgfsys@moveto{10pt}{10pt}
% \pgfsys@lineto{0pt}{0pt}
% \pgfsys@stroke
%
% This will draw a line from (10pt,10pt) to the origin of the picture.

\def\pgfsys@lineto#1#2{\pgf@sys@fail{path constructions}}
% Continue the current path to (#1,#2) with a line.

\def\pgfsys@curveto#1#2#3#4#5#6{\pgf@sys@fail{path constructions}}
% Continue the current path with a bezier curve to (#5,#6). The
% control points of the curve are at (#1,#2) and (#3,#4).

\def\pgfsys@rect#1#2#3#4{\pgf@sys@fail{path constructions}}
% Append a rectangle to the current path whose lower left corner is at
% (#1,#2) and whose width/height is given by (#3,#4).
%
% This command can be `mapped back' to moveto and lineto commands, but
% it is included since pdf has a special, quick version of this command.

\def\pgfsys@closepath{\pgf@sys@fail{path constructions}}
% Close the current path. This results in joining the current point of
% the path with the point specified by the last moveto
% operation. Typically, this is preferable over using lineto to the
% last point specified by a moveto, since the line starting at this
% point and the line ending at this point will be smoothly joined by
% closepath.
%
% Example:
%
% \pgfsys@moveto{0}{0}
% \pgfsys@lineto{10}{10}
% \pgfsys@lineto{0}{10}
% \pgfsys@closepath
% \pgfsys@stroke
%
% versus
%
% \pgfsys@moveto{0}{0}
% \pgfsys@lineto{10}{10}
% \pgfsys@lineto{0}{10}
% \pgfsys@lineto{0}{0}
% \pgfsys@stroke
%
% The difference between the above will be that in the second triangle
% the corner at the origin will be wrong; it will just be the overlay
% of two lines going in different directions, not a sharp pointed
% corner.






%
%
% Stroking and filling commands
%
%

\def\pgfsys@stroke{\pgf@sys@fail{path usage}}
% Stroke the current path (as if it were drawn with a pen). A number
% of so-called graphic state parameters influence this, which can be
% set in different ways:
%
% - linewidth
%   The `thickness' of the line. A width of 0 is the thinnest width
%   renderable on the device. On a high-resolution printer this may
%   become invisible and should be avoided. A good choice is 0.4pt,
%   which is the default.
%
% - stroke color
%   This special color is used for stroking. If it is not set, the
%   current color is used.
%
% - cap
%   The cap describes how the endings of lines a drawn. A round cap
%   adds a little half circle to these endings. A butt cap ends the
%   lines exactly at the end (or start) point without anything
%   added. A rectangular cap ends the lines like the butt cap, but the
%   lines protrude over the endpoint by the line thickness. (See also
%   the pdf or postscript manual). If the path has been closed, no cap
%   is drawn.
%
% - join
%   This describes how a bend (a join) in a path is rendered. A round
%   join draws bends using small arcs. A bevel join just draws the two
%   lines and then fills the join minimally so that it becomes
%   convex. A miter join extends the lines so that they form a single
%   sharp corner, but only up to a certain miter limit. (See the pdf
%   or postscript manual once more).
%
% - dash
%   The line may be dashed according to a dashing pattern.
%
% - clipping area
%   If a clipping area is established, only those parts of the path
%   that are inside the clipping area will be drawn.
%
% In addition to stroking a path, the path may also be used for
% clipping after it has been stroked. This will happen if the
% \pgfsys@clipnext is used prior to this command, see there for
% details.

\def\pgfsys@fill{\pgf@sys@fail{path usage}}
% This command fills the area surrounded by the current path. If the
% path has not yet been closed, it is closed prior to filling. The
% path itself is not stroked. For self-intersecting paths or paths
% consisting of multiple parts, the nonzero winding number rule is
% used to determine whether a point is inside or outside the
% path, except if \ifpgfsys@eorule holds -- in which case the even-odd
% rule should be used. (See the pdf or postscript manual for details.)
%
% The following graphic state parameters influence the filling:
%
% - fill color
%   If the fill color is not especially set, the current color is
%   used.
%
% - clipping area
%   If a clipping area is established, only those parts of the filling
%   area that are inside the clipping area will be drawn.
%
% In addition to filling the path, the path will also be used for
% clipping if \pgfsys@clipnext is used prior to this command.

\def\pgfsys@fillstroke{\pgf@sys@fail{path usage}}
% First, the path is filled, then the path is stroked. If the fill and
% stroke colors are the same (or if they are not specified and the
% current color is used), this yields almost the same as a
% \pgfsys@fill. However, due to the line thickness of the stroked
% path, the fillstroked area will be slightly larger.
%
% In addition to stroking and filling the path, the path will also be
% used for clipping if \pgfsys@clipnext is used prior to this command.

\def\pgfsys@clipnext{\pgf@sys@fail{path usage}}
% This command should be issued after a path has been constructed, but
% before it has been stroked and/or filled or discarded. When the
% command is used, the next stroking/filling/discarding command will
% first be executed normally. Then, afterwards, the just-used path
% will be used for subsequent clipping. If there has already been a
% clipping region, this region is intersected with the new clipping
% path (the clipping cannot get bigger). The nonzero winding number
% rule is used to determine whether a point is inside or outside the
% clipping area or the even-odd rule, depending on whether
% \ifpgfsys@eorule holds.

\def\pgfsys@discardpath{\pgf@sys@fail{path usage}}
% Normally, this command should `throw away' the current path.
% However, after \pgfsys@clipnext has been called, the current path
% should subsequently be used for clipping. See \pgfsys@clipnext for
% details.

\def\pgfsys@closestroke{\pgfsys@closepath\pgfsys@stroke}
% This command should have the same effect as first closing the path
% and then stroking it.

\newif\ifpgfsys@eorule
% Decides, whether even-odd filling and clipping is
% to be used or normal the non-zero winding number rule. See the
% pdf-documentation for details on what these rules are.




%
%
% Stroking options
%
%
\def\pgfsys@setlinewidth#1{\pgf@sys@fail{setting the line width}}
% Sets the width of lines, when stroked, to #1, which must be a TeX
% dimension (as text).

\def\pgfsys@buttcap{\pgf@sys@fail{setting the line cap}}
% Sets the cap to a butt cap. See \pgfsys@stroke.

\def\pgfsys@roundcap{\pgf@sys@fail{setting the line cap}}
% Sets the cap to a round cap. See \pgfsys@stroke.

\def\pgfsys@rectcap{\pgf@sys@fail{setting the line cap}}
% Sets the cap to a rectangular cap. See \pgfsys@stroke.

\def\pgfsys@miterjoin{\pgf@sys@fail{setting the line join}}
% Sets the join to a miter join. See \pgfsys@stroke.

\def\pgfsys@setmiterlimit#1{\pgf@sys@fail{setting the line cap}}
% Sets the miter limit of lines to #1. See the pdf or postscript
% documentation.

\def\pgfsys@roundjoin{\pgf@sys@fail{setting the line cap}}
% Sets the join to a round join. See \pgfsys@stroke.

\def\pgfsys@beveljoin{\pgf@sys@fail{setting the line cap}}
% Sets the join to a bevel join. See \pgfsys@stroke.

\def\pgfsys@setdash#1#2{\pgf@sys@fail{setting the dashing pattern}}
% Sets the dashing patter. #1 should be a list of lengths separated by
% commas. #2 should be a single dimension.
%
% The list of values in #1 is used to determine the lengths of the
% `on' phases of the dashing and of the `off' phases. For example, if
% #1 is `3pt,4pt', then the dashing pattern is `3pt on followed by 4pt
% off, followed by 3pt on, followed by 4pt off, and so on'. A pattern
% of `.5pt,4pt,3pt,1.5pt' means `.5pt on, 4pt off, 3pt on, 1.5pt off, .5pt on,
% ...'. If the number of entries is odd, the last one is used twice,
% so `3pt' means `3pt on, 3pt off, 3pt on, 3pt off,...'. An empty list
% means `always on'.
%
% The second argument determines the `phase' of the pattern. For
% example, for a pattern of `3pt,4pt' and a phase of `1pt', the pattern
% would start: `2pt on, 4pt off, 3pt on, 4pt off, 3pt on, 4pt off...'.





%
%
% Color stuff
%
%

\def\pgfsys@color@rgb@stroke#1#2#3{\pgf@sys@fail{color}}
% Sets the color used for stroking operations to the given rgb tuple
% (numbers between 0 and 1).
%
% Example: Make stroked text dark red: \pgfsys@color@rgb@stroke{0.5}{0}{0}
%
% The special stroking color is only used if the stroking color has
% been set since the last \pgfutil@color command. Thus, each \pgfutil@color command
% should reset both the stroking and filling colors.

\def\pgfsys@color@rgb@fill#1#2#3{\pgf@sys@fail{color}}
% Sets the color used for filling operations to the given rgb tuple
% (numbers between 0 and 1). This color may be different from the
% stroking color.

\def\pgfsys@color@rgb#1#2#3{\pgfsys@color@rgb@stroke{#1}{#2}{#3}\pgfsys@color@rgb@fill{#1}{#2}{#3}}
% Sets the color used for filling and stroking operations. This is a
% ``low-level'' version of the \pgfutil@color command. It is currently only
% used by the plain tex version to support colors.

\def\pgfsys@color@cmyk@stroke#1#2#3#4{\pgf@sys@fail{color}}
% Sets the color used for stroking operations to the given cymk tuple
% (numbers between 0 and 1).

\def\pgfsys@color@cmyk@fill#1#2#3#4{\pgf@sys@fail{color}}
% Sets the color used for filling operations to the given cymk tuple
% (numbers between 0 and 1).

\def\pgfsys@color@cmyk#1#2#3#4{\pgfsys@color@cmyk@stroke{#1}{#2}{#3}{#4}\pgfsys@color@cmyk@fill{#1}{#2}{#3}{#4}}
% Sets the color used for filling and stroking operations.

\def\pgfsys@color@cmy@stroke#1#2#3{\pgf@sys@fail{color}}
% Sets the color used for stroking operations to the given cym tuple
% (numbers between 0 and 1).

\def\pgfsys@color@cmy@fill#1#2#3{\pgf@sys@fail{color}}
% Sets the color used for filling operations to the given cym tuple
% (numbers between 0 and 1).

\def\pgfsys@color@cmy#1#2#3{\pgfsys@color@cmy@stroke{#1}{#2}{#3}\pgfsys@color@cmy@fill{#1}{#2}{#3}}
% Sets the color used for filling and stroking operations.

\def\pgfsys@color@gray@stroke#1{\pgf@sys@fail{color}}
% Sets the color used for stroking operations to the given gray level
% (number between 0 and 1).

\def\pgfsys@color@gray@fill#1{\pgf@sys@fail{color}}
% Sets the color used for filling operations to the given gray level
% (number between 0 and 1).

\def\pgfsys@color@gray#1{\pgfsys@color@gray@stroke{#1}\pgfsys@color@gray@fill{#1}}
% Sets the color used for filling and stroking operations.

\newif\ifpgfsys@color@reset@inorder
% This if decides whether the color resetting of the
% \pgfsys@color@reset command may assume that all color commands are
% given ``in order.'' This means that this command will not be called,
% say, at the beginning of a box that is later inserted at a point
% where a fill color was actively set. This may allow the command to
% produce more compact code when it ``knows'' that no special fill or
% stroke color has been set.
\pgfsys@color@reset@inordertrue


\def\pgfsys@color@reset{}
% This command will be called when the \pgfutil@color command is used. It
% should purge any internal settings of stroking and filling
% color. After this call, till the next use of a command like
% \pgfsys@color@rgb@fill, the current color installed by the \pgfutil@color
% command should be used.

\def\pgfsys@color@unstacked#1{}
% This slightly obscure command causes the color stack to be
% tricked. When called, this command should set the current color to
% #1 without causing any change in the color stack.

\def\pgfsys@text@to@black@hook{}
% Another slightly obscure command that allows tikz to hack into
% dvisvgm's color management inside text nodes. The problem is that
% the color "black" is special for dvisvgm.



%
%
% Coordinate system transformations
%
%

\def\pgfsys@transformcm#1#2#3#4#5#6{\pgf@sys@fail{transformations}}
% Perform a concatenation of the low-level current transformation
% matrix with the matrix given by the values #1 to #6. The
% transformation matrix is a transformation on a homogeneous
% 2D-coordinate system. See the pdf or postscript manual for details.

\def\pgfsys@transformshift#1#2{\pgfsys@transformcm{1}{0}{0}{1}{#1}{#2}}
% This command will change the origin of the low-level coordinate
% system to (#1,#2).

\def\pgfsys@transformxyscale#1#2{\pgfsys@transformcm{#1}{0}{0}{#2}{0bp}{0bp}}
% This command will scale the low-level coordinate system (and
% everything that is drawn) by a factor of #1 in x-directed and #2 in
% y-direction. Note that this applies to everything, including
% lines. So a scaled line will have a different width and may even
% have a different width when going along the x-axis and when going
% along the y-axis, if the scaling is different in these directions.

\def\pgfsys@viewboxmeet#1#2#3#4#5#6#7#8{\pgfsys@beginscope\pgf@sys@default@viewbox@impl{#1}{#2}{#3}{#4}{#5}{#6}{#7}{#8}{<}}
% Consider the two rectangles R with lower left corner
% (#5,#6) and upper right (#7,#8) and S with lower left corner
% (#1,#2) and upper right (#3,#4). Now, consider the transformation
% that scales and translates R so that the result T has the same
% center as S and has maximal size while still being inside S. This
% transformation gets installed in a new graphics group, which must be
% ended with a corresponding \pgfsys@endviewbox.

\def\pgfsys@viewboxslice#1#2#3#4#5#6#7#8{\pgfsys@beginscope\pgf@sys@default@viewbox@impl{#1}{#2}{#3}{#4}{#5}{#6}{#7}{#8}{>}}
% Works like \pgfsys@viewboxmeet, but R3 will have minimal size so
% that in contains all of R2.

\def\pgfsys@endviewbox{\pgfsys@endscope}
% Ends a viewbox scope started by either of the previous two commands.


\def\pgf@sys@default@viewbox@impl#1#2#3#4#5#6#7#8#9{{%
    \pgf@x#1%
    \pgf@y#2%
    \pgf@xa#3%
    \pgf@ya#4%
    \advance\pgf@xa by-\pgf@x%
    \advance\pgf@ya by-\pgf@y%
    \pgfmathdivide@{\pgf@sys@tonumber\pgf@xa}{\pgf@sys@tonumber\pgf@ya}%
    \let\aspectr\pgfmathresult%
    \pgf@xb#5%
    \pgf@yb#6%
    \pgf@xc#7%
    \pgf@yc#8%
    \advance\pgf@xc by-\pgf@xb%
    \advance\pgf@yc by-\pgf@yb%
    \pgfmathdivide@{\pgf@sys@tonumber\pgf@xc}{\pgf@sys@tonumber\pgf@yc}%
    \let\aspects\pgfmathresult%
    \ifdim\aspectr pt#9\aspects pt%
      \pgfmathdivide@{\pgf@sys@tonumber\pgf@xa}{\pgf@sys@tonumber\pgf@xc}%
    \else%
      \pgfmathdivide@{\pgf@sys@tonumber\pgf@ya}{\pgf@sys@tonumber\pgf@yc}%
    \fi%
    \advance\pgf@x by.5\pgf@xa%
    \advance\pgf@xb by.5\pgf@xc%
    \advance\pgf@y by.5\pgf@ya%
    \advance\pgf@yb by.5\pgf@yc%
    \pgfsys@transformcm{\pgfmathresult}{0}{0}{\pgfmathresult}{\pgf@x}{\pgf@y}%
    \pgfsys@transformcm{1}{0}{0}{1}{-\pgf@xb}{-\pgf@yb}%
  }%
}




%
%
% Graphic state saving and restoring.
%
%

\def\pgfsys@beginscope{\pgf@sys@fail{scoping}}
% Saves the current graphic state on a graphic state stack. All
% changes to the graphic state parameters mentioned for \pgfsys@stroke
% and \pgfsys@fill will be local to the current graphic state and will
% the old values will be restored after endscope is used.
%
% WARNING: pdf and postscript differ with respect to the question of
% whether the current path is part of the graphic state or not. For
% this reason, you should never use this command unless the path is
% currently empty. For example, it might be a good idea to use
% discardpath prior to calling this command.

\def\pgfsys@endscope{\pgf@sys@fail{scoping}}
% Restores the last saved graphic state.

\def\pgfsys@begin@idscope{%
  \begingroup%
    \edef\pgf@sys@cacheref{\pgfsys@id@refcurrent}%
    \expandafter\let\expandafter\pgfsys@beg@save\csname pgf@sys@att@beg@\pgf@sys@cacheref\endcsname%
    \expandafter\let\expandafter\pgfsys@end@save\csname pgf@sys@att@end@\pgf@sys@cacheref\endcsname%
    \ifx\pgfsys@beg@save\relax%
      \ifx\pgfsys@end@save\relax%
      \else%
        \pgfsys@call@save%
      \fi%
    \else%
      \pgfsys@call@save%
    \fi%
    \pgfsys@invalidate@currentid%
      \begingroup%
}
% Starts an "id scope" where the current value of \pgfsys@use@id and
% \pgfsys@use@type is used. If the combination has been used before,
% nothing happens. Otherwise, depending of the driver, a graphic
% scope may or may not be created. To ensure that a graphic scope is
% created, use \pgfsys@beginscope additionally.

\def\pgfsys@call@save{%
  \pgfsys@beginscope%
  \pgfsys@beg@save%
  \expandafter\global\expandafter\let\csname pgf@sys@att@beg@\pgfsys@id@refcurrent\endcsname\relax%
  \expandafter\global\expandafter\let\csname pgf@sys@att@end@\pgfsys@id@refcurrent\endcsname\relax%
}

\def\pgfsys@end@idscope{
      \endgroup%
    \ifx\pgfsys@beg@save\relax%
      \ifx\pgfsys@end@save\relax%
      \else%
        \pgfsys@call@end%
      \fi%
    \else%
      \pgfsys@call@end%
    \fi%
  \endgroup
}
% Ends an id scope.

\def\pgfsys@call@end{%
  \pgfsys@end@save%
  \pgfsys@endscope%
}


\def\pgfsys@begin@text{}
% Inside a text box you should not use graphics command. However, TikZ
% needs to insert some color commands at the beginnings of text
% boxes. For most drivers, this is not a problem; except for SVG,
% where every change of the graphics state opens a scope and needs a
% closing scope. This command should be used at the beginning of text
% boxes containing such color commands and the matching
% \pgfsys@end@text at the end. If the box does not contain commands
% that change the graphics state, this command is not needed.

\def\pgfsys@end@text{}
% Closes a \pgfsys@begin@text.



%
%
% Id management (all of these commands have a default implementation
% any typically need not be redefined by a driver)
%
%

% The id count
\newcount\pgf@sys@id@count

\def\pgfsys@new@id#1{%
  \edef#1{pgf\the\pgf@sys@id@count}%
  \global\advance\pgf@sys@id@count by1\relax%
}
% Creates a new id and stores it in the macro #1.

\def\pgfsys@use@id#1{%
  \edef\pgf@sys@id@current@id{#1}%
  \let\pgfsys@current@type\pgfutil@empty%
}
\let\pgf@sys@id@current@id\pgfutil@empty
% Sets the id to be used for the next graphic object (group, box, or
% path). The type is reset to be empty. #1 must be a macro name that
% has previously been set using \pgfsys@new@id.
%
% Each combination of id and type can be used only once, on a second
% use nothing happens.

\def\pgfsys@clear@id{%
  \let\pgf@sys@id@current@id\pgfutil@empty%
}
% Clears the current id, so nothing can be referenced.

\def\pgfsys@use@type#1{%
  \edef\pgfsys@current@type{#1}%
  \pgfsys@register@type\pgfsys@current@type%
}
\let\pgfsys@current@type\pgfutil@empty
% Sets the type to be used for the next graphic object.

\def\pgfsys@append@type#1{%
  \ifx\pgfsys@current@type\pgfutil@empty%
    \pgfsys@use@type{#1}%
  \else%
    \pgfsys@use@type{\pgfsys@current@type.#1}%
  \fi%
}
% Appends something to the current type

\def\pgfsys@register@type#1{%
  \expandafter\let\expandafter\pgf@sys@temp\csname pgf@sys@reg@type@#1\endcsname%
  \ifx\pgf@sys@temp\relax%
    {%
      \c@pgf@counta\pgf@sys@type@count\relax%
      \global\advance\c@pgf@counta by1\relax%
      \edef\pgf@sys@type@count{\the\c@pgf@counta}%
      \expandafter\xdef\csname pgf@sys@reg@type@#1\endcsname{y\the\c@pgf@counta}%
    }%
  \fi%
}
% Registers a type with the system. Must be called before any use of
% the type
\def\pgf@sys@reg@type@{}
\def\pgf@sys@reg@type@background{b}
\def\pgf@sys@reg@type@path{p}
\def\pgf@sys@reg@type@text{t}
\expandafter\def\csname pgf@sys@reg@type@background.path\endcsname{bp}
\def\pgf@sys@type@count{0}

\def\pgfsys@push@type{%
  \expandafter\expandafter\expandafter\def\expandafter\expandafter\expandafter\pgf@sys@typestack%
  \expandafter\expandafter\expandafter{\expandafter\expandafter\expandafter\def\expandafter\expandafter\expandafter\pgfsys@current@type%
    \expandafter\expandafter\expandafter{\expandafter\pgfsys@current@type\expandafter}%
    \expandafter\def\expandafter\pgf@sys@typestack\expandafter{\pgf@sys@typestack}}%
}
\let\pgf@sys@typestack\pgfutil@empty
% Pushes the current type onto a global stack. This is useful for
% temporarily changing the type without creating \TeX scopes.

\def\pgfsys@pop@type{\pgf@sys@typestack}
% Pops the last id from the stack.

\def\pgfsys@id@ref#1#2{#1\csname pgf@sys@reg@type@#2\endcsname}
% Expands to a text that can be inserted as a reference. #1 must be a
% reference created \pgfsys@new@id, #2 must be a type that has been
% registered using \pgfsys@id@register@type.

\def\pgfsys@id@refcurrent{\pgfsys@id@ref{\pgf@sys@id@current@id}{\pgfsys@current@type}}
% Expands to a text that can be inserted as a reference to the current
% id-type pair in use.

\def\pgfsys@invalidate@currentid{%
  \expandafter\global\expandafter\let\csname pgf@sys@id@keylist@\pgfsys@id@refcurrent\endcsname\pgfutil@empty%
}
% Mark the current id-type pair as used.

\def\pgfsys@attach@to@id#1#2#3#4{%
  \pgfsys@register@type{#2}%
  \expandafter\def\expandafter\pgf@sys@tempbeg\expandafter{\csname pgf@sys@att@beg@\pgfsys@id@ref{#1}{#2}\endcsname}%
  \expandafter\def\expandafter\pgf@sys@tempend\expandafter{\csname pgf@sys@att@end@\pgfsys@id@ref{#1}{#2}\endcsname}%
  \expandafter\ifx\pgf@sys@tempbeg\relax%
    \expandafter\let\pgf@sys@tempbeg\pgfutil@empty%
  \fi%
  \expandafter\ifx\pgf@sys@tempend\relax%
    \expandafter\let\pgf@sys@tempend\pgfutil@empty%
  \fi%
  \expandafter\let\expandafter\pgf@sys@tempbeg@cont\pgf@sys@tempbeg%
  \expandafter\let\expandafter\pgf@sys@tempend@cont\pgf@sys@tempend%
  \expandafter\expandafter\expandafter\gdef\expandafter\pgf@sys@tempbeg\expandafter{\pgf@sys@tempbeg@cont#3}%
  \def\pgf@sys@temp{#4}%
  \expandafter\expandafter\expandafter\expandafter\expandafter\expandafter\expandafter\gdef\expandafter\expandafter\expandafter\pgf@sys@tempend\expandafter\expandafter\expandafter{\expandafter\pgf@sys@temp\pgf@sys@tempend@cont}%
}
% Attaches some code to an id-type pair so that when the id's scope
% gets created, #3 is added at the beginning and #4 is added at the
% end.

\def\pgfsys@if@fresh@id#1#2#3#4{%
  \edef\pgf@sys@temp{#1}%
  \ifx\pgf@sys@temp\pgfutil@empty% empty id always counts as "has been used"
    #4%
  \else%
    \pgfsys@register@type{#2}%
    \expandafter\ifx\csname pgf@sys@id@keylist@\pgfsys@id@ref{#1}{#2}\endcsname\pgfutil@empty%
      % has already been used!
      #4%
    \else%
      #3%
    \fi%
  \fi%
}
% Checks whether a given id-type pair #1-#2 has not been used, yet. If
% so, #3 is executed, otherwise #4.

\def\pgfsys@if@fresh@currentid#1#2{%
  \ifx\pgf@sys@id@current@id\pgfutil@empty% empty id always counts as "has been used"
    #2%
  \else%
    \expandafter\ifx\csname pgf@sys@id@keylist@\pgfsys@id@refcurrent\endcsname\pgfutil@empty%
      % has already been used!
      #2%
    \else%
      #1%
    \fi%
  \fi%
}
% Checks whether the current id-type pair in use has not been used,
% yet. If so, #1 is executed, otherwise #2.




%
%
% Special commands
%
%

\def\pgfsys@imagesuffixlist{}
% A list of suffixes, separated by `:', that will be tried when
% searching for an image.



%
%
% Transparency
%
%

\def\pgfsys@opacity#1{\pgfsys@stroke@opacity{#1}\pgfsys@fill@opacity{#1}}
% Ideally, this should sets the opacity of all operations, treating
% the group as a transparency group with 1 meaning fully opaque, 0
% meaning fully transparent. Since most drivers do not support this,
% the fallback is to set the stroke and fill opacity at the same
% time. This is not quite the same, however.

\def\pgfsys@stroke@opacity#1{\pgf@sys@fail{opacity}}
% Sets the opacity of stroking operations. 1 means fully opaque, 0
% means fully transparent.

\def\pgfsys@fill@opacity#1{\pgf@sys@fail{opacity}}
% Sets the opacity of filling operations. 1 means fully opaque, 0
% means fully transparent.

\def\pgfsys@transparencygroupfrombox#1{\pgf@sys@fail{transparency groups}}
% This takes a TeX box and converts it into a transparency
% group. This means that any transparency settings apply to the box as
% a whole. For instance, if a box contains two overlapping black
% circles and you draw them normally with 50% transparency, then the
% overlap will be darker than the rest. By comparison, if the circles
% are part of a transparency group, the overlap will get the same
% color as the rest.
%
% The following settings modify the behaviour of a transparency group:

\newif\ifpgfsys@transparency@group@isolated
\newif\ifpgfsys@transparency@group@knockout


\def\pgfsys@blend@mode#1{\pgf@sys@fail{blend mode}}
% Sets the current blend mode. See the PDF specification for details
% on blend modes.


\def\pgfsys@definemask{\pgf@sys@fail{image masks}}
% This command declares a mask for usage with images. It works similar
% to \pgfsys@defineimage: Certain macros are set when the command is
% called. The result should be to set the macro \pgf@mask to a pdf
% object count that can subsequently be used as a soft mask. The
% following macros will be set when this command is invoked:
%
% - \pgf@filename
%   File name of the mask to be defined.
%
% - \pgf@maskmatte
%   The so-called matte of the mask (see the pdf documentation for
%   details). The matte is a color specification consisting of 1, 3 or
%   4 numbers between 0 and 1. The number of numbers depends on the
%   number of color channels in the image (not in the mask!). It will
%   be assumed that the image has been preblended with this color.

\def\pgfsys@fadingfrombox#1#2{\pgf@sys@fail{fadings}}
% Declares an fading named #1. The parameter #2 must be a
% box. Its contents luminosity determines the opacity of the resulting
% mask.

\def\pgfsys@usefading#1#2#3#4#5#6#7{\pgf@sys@fail{fadings}}
% Installs a previously declared fading #1 in the current graphics
% state. Parameters #2 to #7 specify a transformation cm that should
% be applied to the mask before it is used.

\def\pgfsys@clipfading{}
% This default implementation is a no-op for all drivers except `dvips'
% where we have to set the clipping path _before_ installing the fading (used in
% \pgfsetfadingforcurrentpath, \pgfsetfadingforcurrentpathstroked commands of
% the basic layer)


%
%
% Shading commands
%
%

\def\pgfsys@horishading#1#2#3{\pgf@sys@fail{shadings}\pgf@sys@noshading{#1}}
% Declares a horizontal shading for later use. The effect of this
% command should be the definition of a macro called `@pgfshading#1!'
% (or \csname @pdfshading#1!\endcsname, to be precise). When invoked,
% this new macro should insert a shading at the current position.
%
% #1 is the name of the shading, which is also used in the output
% macro name. #2 is the height of the shading and must be given as a
% TeX dimension like `2cm' or `10pt'. #3 is a shading color
% specification as specified in the pgfuserguide. The shading
% specification implicitly fixes the width of the shading.
%
% When `@pgfshading#1!' is invoked, it should insert a box of height
% #2 and the width implicit in the shading declaration.

\def\pgfsys@vertshading#1#2#3{\pgf@sys@fail{shadings}\pgf@sys@noshading{#1}}
% Like horishading, only for vertical shadings. This time, the height
% of the shading is implicit in #3 and the width is given as #2.

\def\pgfsys@radialshading#1#2#3{\pgf@sys@fail{shadings}\pgf@sys@noshading{#1}}
% Declares a radial shading. Like the previous macros, this command
% should setup the macro `@pgfshading#1!', which upon invocation
% should insert a radial shading whose size is implicit in #3.
%
% The parameter #2 is a pgf point specification if the starting point
% of the shading. See the pgfuserguide for details.

\def\pgfsys@functionalshading#1#2#3#4{\pgf@sys@fail{shadings (functional)}\pgf@sys@noshading{#1}}
% Declares a shading using a PostScript-like function that provides a
% color for each point. Like the previous macros, this command
% should setup the macro `@pgfshading#1!' so that it will produce a
% box containing the desired shading.
%
% Parameter #1 is the name of the shading. Parameter #4 is a
% Postscript-like function (type 4 function of the PDF specification)
% as described in Section 3.9.4 of the PDF Specification version 1.7.
% Parameter #2 and #3 are pgf point specifications that specifies the
% lower left and upper right coordinate.
%
% When function #4 is evaluated, the coordinate of the current
% point will be on the (virtual) PostScript stack in bp units. After
% the function has been evaluated, the stack should consist of three
% numbers (not integers! - the Apple PDF renderer is broken in this
% regard, so add cvr's at the end if needed) that represent the red,
% green, and blue components of the color.
%
% A buggy #4 will result is *totally unpredictable chaos* during
% rendering.

\def\pgf@sys@noshading#1{\expandafter\global\expandafter\let\csname @pgfshading#1!\endcsname=\pgfutil@empty}




%
%
% Pattern stuff
%
%

\def\pgfsys@declarepattern#1#2#3#4#5#6#7#8#9{\pgf@sys@fail{patterns}}
% This command declares a new colored or uncolored pattern. Uncolored
% patterns have no inherent color, the color is provided when they are
% set. Colored patters have an inherent color.
% #1       = name of the pattern (must be a plain suffix for use in
%            postscript/pdf/etc.)
% #2 to #5 = four coordinates describing the corners of a bounding box
%            of pattern cells
% #6       = x-step
% #7       = y-step
% #8       = pattern code
% #9       = 0 if uncolored patter, 1 if colored pattern.

\def\pgfsys@setpatternuncolored#1#2#3#4{\pgf@sys@fail{patterns}}
% Sets an uncolored pattern #1 as filling color. The numbers #2, #3,
% #4 must be an rgb-value to be used as the color of the pattern.

\def\pgfsys@setpatterncolored#1{\pgf@sys@fail{patterns}}
% Sets an colored pattern #1 as filling color.





%
%
% Animations
%
%


\newif\ifpgfsysanimationsupported
% Will be set to true by a driver when animations are supported. Note
% that snapshots are always supported.


\def\pgfsys@animate#1{\pgf@sys@fail{animations}}
% Adds an animation of the attribute #1 to the current scope.
% The configuration of the animation (like start times, etc.) is taken
% from the current values of the animation keys, which are configured
% using calls to \pgfsys@animation@..., which setup an animation
% key for the local scope. This means that you typically setup these
% keys in a scope and at the end of the scope call \pgfsys@animate.
%
% Some keys accumulate (like begin or keytime), most do not. In those
% cases, the last call overrides the previous one.
%
% Not all value types make sense or are allowed, it depends on the
% specific \pgfsys@animate call.
%
% Example:
%
% {
%   \pgfsys@animation@offset{10}{begin}%
%   \pgfsys@animation@event{}{click}{1}{begin}%
%   \pgfsys@animation@time{5}{1}{1}{0}{0}
%   \pgfsys@animation@val@scalar{1}
%   \pgfsys@animation@time{8}{1}{1}{0}{0}
%   \pgfsys@animation@val@scalar{0.5}
%   \pgfsys@animation@sum%
%   \pgfsys@animate{fillopacity}%
% }
%
% When a driver does not support a specific attribute #1, it should
% issue a \pgf@sys@fail. Typical attributes are things like "opacity"
% or "linewidth" or "dash".
%
% The following values are permissible for "#1":
%
% "opacity" adds an animation of the opacity.
% Type: scalar
%
% "fillopacity" adds an animation of the fill opacity.
% Type: scalar
%
% "strokeopacity" adds an animation of the stroke opacity.
% Type: scalar
%
% "visibility" adds an animation of the visibility.
% Type: text ("visible" or "hidden")
%
% "linewidth" adds an animation of the line width.
% Type: dimension
%
% "dash" adds an animation of the dash.
% Type: dash
%
% "translate" adds an animation of the transformation.
% Type: translate
%
% "scale" adds an animation of the transformation.
% Type: scale
%
% "rotate" adds an animation of the transformation.
% Type: scalar
%
% "skewx" adds an animation of the transformation.
% Type: scalar
%
% "skewy" adds an animation of the transformation.
% Type: scalar
%
% "motion" adds a movement along a path.
% Type: motion
%
% "strokecolor" adds an animation of the stroke color.
% Type: color
%
% "fillcolor" adds an animation of the fill color.
% Type: color
%
% "viewbox" adds an animation of the view box.
% Type: view box
%
% "path" adds an animation of the path to the path specified by
% \pgfsys@animation@whom
% Type: path
%
% "syncbase" adds a sync base to the animation. A sync base does not animate
% anything by itself, but other animations can begin and end relative
% to it.
%
% "none" suppresses the animation
% Type: any




% Animation timing :
%
% You specify an animation using a timeline. A timeline consists of a
% sequence of times together with values for these times. The times
% must be given in increasing order. You use the command
% \pgfsys@animation@time to specify a time and
% \pgfsys@animation@val@... to specify a value. The time must be
% specified first, the time--value is created when the value command is
% used.
%
% Times are given in seconds (so 0.01 is one hundredth of a second).
%
% The animation's timeline will start with the first time mentioned
% (not necessarily zero and not necessarily a positive value) and ends
% with the last value. The duration is the difference between these
% two values.

\def\pgfsys@animation@restart@always{}
% Specifies that the animation can always restart

\def\pgfsys@animation@restart@never{}
% Specifies that the animation should never restart

\def\pgfsys@animation@restart@whennotactive{}
% Specifies that the animation should only restart when it is not active

\def\pgfsys@animation@repeat#1{}
% #1 must be a number

\def\pgfsys@animation@repeat@indefinite{}
% Repeats forever

\def\pgfsys@animation@repeat@dur#1{}
% #1 must be a clock value

\def\pgfsys@animation@freezeatend{}
% Specifies that the last value of the animation remains in force at
% the end

\def\pgfsys@animation@removeatend{}
% Specifies that the effect of the animation is removed at the end

\def\pgfsys@animation@time#1#2#3#4#5{}
% Specifies a time together with spline control points for the entry
% and the exit to this time. The value #1 is a time in seconds (so 60
% means 1 minute, 0.001 means one millisecond). The values #2 and #3
% specify the control point of the spline *entering* the time point,
% while the values #4 and #5 specify the control point of the spline
% *leaving* the time point. They are all dimensionless values between
% 0 and 1. For a linear animation, set the first two to 1 and the last
% two to 0.
%
% Example:
% \pgfsys@animation@time{5}{1}{1}{0.25}{0.1}
% \pgfsys@animation@val@scalar{80}
% \pgfsys@animation@time{8}{.75}{1}{0}{0}
% \pgfsys@animation@val@scalar{90}
%
% In SVG, this will create the following entries:
% begin     = +5s
% duration  = 3s
% keyTimes  = 0;1
% values    = 80;90
% keySpline = 0.25 0.1 0.75 1
%
% For the key spline, note that the entry spline controls of the first
% time and the exit spline controls of the last point are dropped.
%
%
% When the exit spline takes the special values #4="stay" and #5="0",
% the attribute's value "stays" until the next value for the next time
% (it "jumps" to the next value then). This corresponds, roughly, to
% an "infinite" #4.
%
% Similarly, when the entry spline takes the special value #2="jump"
% and #3="1", the value immediately jumps from the previous value to
% the next value when the previous value was "created".

\def\pgfsys@stay@text{{stay}{0}}
\def\pgfsys@jump@text{{jump}{1}}

\def\pgfsys@animation@base{}
% Think of this command as setting the value of the attribute in the
% timeline before the timeline starts. Normally, an animation only
% influences the timeline while the animation is running (which may be
% forever) and before and after the animation is running, the
% attribute's value is not modified by the animation (so, it is just
% inherited from the surrounding scope). When a base value is set,
% while the animation is not running, this value is used. For most
% attributes you can achieve the same effect by surrounding the
% to-be-animated object by a scope and setting the value there, but
% using bases is often more convenient and, for instance for the
% visibility attribute, it is necessary to use a base.
%
% Note that when a backend driver does not support animations, base
% values will not be depicted unless a snapshot is used explicitly.

\def\pgfsys@animation@offset#1#2{}
% #1 must be a value in seconds, #2 is the target "begin" or "end" to
% which this offset is appended.

\def\pgfsys@animation@syncbegin#1#2#3#4{}
% #1 is the id of a sync base (obtained by
% \pgfsys@new@id), #2 is the id type (or empty), #3 is an optional
% offset, #4 is the target.

\def\pgfsys@animation@syncend#1#2#3#4{}
% Like the previous command.

\def\pgfsys@animation@event#1#2#3#4#5{}
% #1 is the optional id of another element (obtained by
% \pgfsys@new@id), #2 is the optional type of the id, #3 is an event
% name, #4 is an optional offset, #5 is the target.

\def\pgfsys@animation@repeat@event#1#2#3#4#5{}
% #1 is the optional id of another element, #2 is the type, #3 is a
% repeat count, and #4 is an optional offset, #5 is the target.

\def\pgfsys@animation@accesskey#1#2#3{}
% #1 is a character, #2 is an optional offset, #3 is the target.

\def\pgf@sys@begin@text{begin}


% Accumulation

\def\pgfsys@animation@accumulate{}
% Corresponds to accumulate="sum" in SVG

\def\pgfsys@animation@noaccumulate{}
% Corresponds to accumulate="none" in SVG


% Targeting

\def\pgfsys@animation@whom#1#2{}
% Sets the target of the animation. #1 must be an id previously
% created using \pgfsys@new@id and #2 must be empty or a type
% previously created using \pgfsys@new@id@type


% Attribute definitions

\def\pgfsys@animation@rotatealong{}
% Applicable only to motion animations. Indicates that the
% to-be-animated group should be rotated automatically so that it
% points along the path as time progresses.

\def\pgfsys@animation@norotatealong{}
% Corresponds to rotate="0"

\def\pgfsys@animation@movealong#1{}
% Applicable only to motion animations. #1 is a softpath along which
% the movement will be done.

\def\pgfsys@animation@tip@markers#1#2{}
% Applicable only to path animations. #1 and #2 are markers (declared
% using \pgfsys@marker@declare) that store marker symbols to be added
% at the start and at the end of the to-be-animated path. If empty, no
% markers are added.

\def\pgfsys@animation@canvas@transform#1#2{}
% #1 is some code that may call \pgfsys@transformcm at most once. It
% will be executed when a canvas transformation animation is
% installed for a scope id. #2 is some further codes that may also
% call \pgfsys@transformcm at most once. It will be installed right
% after the animation.
%
% Typically, #1 is some pgf coordinate shift followed by
% {\pgflowlevelsynccm} and #2 is \pgftransforminvert followed by
% \pgflowlevelsynccm.



% Values

\def\pgfsys@animation@val@current{}
% The current value of the attribute. At least with SVG, this can only
% be used for the first value and, then, only if there is only one
% other value.

\def\pgfsys@animation@val@text#1{}
% #1 should be some text.

\def\pgfsys@animation@val@scalar#1{}
% #1 should be a single value.

\def\pgfsys@animation@val@dimension#1{}
% #1 must be a dimension.

\def\pgfsys@animation@val@color@rgb#1#2#3{}
% #1,#2,#3 must be an rgb tuple

\def\pgfsys@animation@val@color@cmyk#1#2#3#4{}
% #1,#2,#3,#4 must be an cmyk tuple

\def\pgfsys@animation@val@color@cmy#1#2#3{}
% #1,#2,#3 must be an cmy tuple

\def\pgfsys@animation@val@color@gray#1{}
% #1 must be a gray value

\def\pgfsys@animation@val@path#1{}
% #1 must consist of path-construction tokens (as returned by the
% softpath layer)

\def\pgfsys@animation@val@translate#1#2{}
% #1,#2 must be dimensions

\def\pgfsys@animation@val@scale#1#2{}
% #1,#2 must be dimensionless

\def\pgfsys@animation@val@viewbox#1#2#3#4{}
% #1,#2 describe the lower left, #3,#4 the upper right corner of the
% view box.

\def\pgfsys@animation@val@dash#1#2{}
% #1,#2 have the syntax of \pgfsys@setdash.



%
%
% Markers
%
%

\def\pgfsys@marker@declare#1#2{%
  {%
    \pgfpicturetrue%
    \pgfsysprotocol@getcurrentprotocol\pgfsys@marker@temp%
    \pgfsysprotocol@setcurrentprotocol\pgfutil@empty%
    \pgfsysprotocol@bufferedtrue%
    \pgfsys@beginscope%
    #2%
    \pgfsys@endscope%
    \xdef#1{\the\pgf@sys@id@count}%
    \expandafter\expandafter\expandafter\global\expandafter\pgfsysprotocol@getcurrentprotocol\csname pgf@sys@marker@prot@#1\endcsname%
    \pgfsys@marker@declare@% curtesy hook
    \pgfsysprotocol@setcurrentprotocol\pgfsys@marker@temp%
    \global\advance\pgf@sys@id@count by1\relax%
  }%
}
\let\pgfsys@marker@declare@\relax
% Declares a new marker symbol whose code is in #2. #1 should be a
% macro name, which will be set to a unique
% value by which the marker can be referenced later on. #2 should be
% code that, when executed, produces recordable code.

\def\pgfsys@marker@use#1{%
  \pgfsysprotocol@literal{\csname pgf@sys@marker@prot@#1\endcsname}%
}
% Draws the marker symbol not at the end of a path, but simply with
% the current transformation matrix






%
%
% RDF
%
%

\def\pgfsys@rdf@about#1{\pgf@sys@fail{rdf}}
% Adds the rdf attribute "about" with value #1 to the next
% idscope. The RDF-spec says: "a SafeCURIEorCURIEorIRI, used for
% stating what the data is about (a 'subject' in RDF terminology);"

\def\pgfsys@rdf@content#1{\pgf@sys@fail{rdf}}
% Adds the rdf attribute content with value #1 to the next
% idscope. The RDF-spec says: "a CDATA string, for supplying
% machine-readable content for a literal (a 'literal object', in RDF
% terminology);"

\def\pgfsys@rdf@datatype#1{\pgf@sys@fail{rdf}}
% Adds the rdf attribute datatype with value #1 to the next
% idscope. The RDF-spec says: "a TERMorCURIEorAbsIRI representing a
% datatype, to express the datatype of a literal;"

\def\pgfsys@rdf@href#1{\pgf@sys@fail{rdf}}
% Adds the rdf attribute href with value #1 to the next idscope
% The RDF-spec says: "a traditionally navigable IRI for
% expressing the partner resource of a relationship (a 'resource
% object', in RDF terminology);"

\def\pgfsys@rdf@inlist{\pgf@sys@fail{rdf}}
% Adds the rdf attribute inlist to the next idscope. The
% RDF-spec says: "An attribute used to indicate that the object
% associated with a rel or property attribute on the same element is
% to be added to the list for that predicate. The value of this
% attribute must be ignored. Presence of this attribute causes a list
% to be created if it does not already exist."

\def\pgfsys@rdf@prefix#1{\pgf@sys@fail{rdf}}
% Adds #1 to the list of the rdf prefix attribute to the next
% idscope. Can be called several times, in which case all values of #1
% are concatenated with whitespaces. The RDF-spec says: "a white space
% separated list of prefix-name IRI pairs of the form NCName ':' ' '+
% xsd:anyURI"

\def\pgfsys@rdf@property#1{\pgf@sys@fail{rdf}}
% Adds #1 to the list of the rdf property attribute to the next
% idscope. Can be called repeatedly. The RDF-spec says: "a white space
% separated list of TERMorCURIEorAbsIRIs, used for expressing
% relationships between a subject and either a resource object if
% given or some literal text (also a 'predicate');"

\def\pgfsys@rdf@rel#1{\pgf@sys@fail{rdf}}
% Adds #1 to the list of the rdf rel attribute to the next
% next idscope. Can be called repeatedly. The
% RDF-spec says: "a white space separated list of
% TERMorCURIEorAbsIRIs, used for expressing relationships between two
% resources ('predicates' in RDF terminology);"

\def\pgfsys@rdf@resource#1{\pgf@sys@fail{rdf}}
% Adds the rdf attribute resource with value #1 to the next idscope. The
% RDF-spec says: "a SafeCURIEorCURIEorIRI for expressing the partner
% resource of a relationship that is not intended to be navigable
% (e.g., a 'clickable' link) (also an 'object');"

\def\pgfsys@rdf@rev#1{\pgf@sys@fail{rdf}}
% Adds #1 to the list of the rdf rev attribute to the next
% idscope. Can be called repeatedly. The
% RDF-spec says: "a white space separated list of
% TERMorCURIEorAbsIRIs, used for expressing reverse relationships
% between two resources (also 'predicates');

\def\pgfsys@rdf@src#1{\pgf@sys@fail{rdf}}
% Adds an rdf attribute src with value #1 to the next idscope. The
% RDF-spec says: "an IRI for expressing the partner resource of a
% relationship when the resource is embedded (also a 'resource
% object');"

\def\pgfsys@rdf@typeof#1{\pgf@sys@fail{rdf}}
% Adds #1 to the list of the rdf typeof attribute to the next
% idscope. Can be called repeatedly. The
% RDF-spec says: "a white space separated list of TERMorCURIEorAbsIRIs
% that indicate the RDF type(s) to associate with a subject;"

\def\pgfsys@rdf@vocab#1{\pgf@sys@fail{rdf}}
% Adds an rdf attribute vocab with value #1 to the next idscope. The
% RDF-spec says: "an IRI that defines the mapping to use when a TERM
% is referenced in an attribute value. See General Use of Terms in
% Attributes and the section on Vocabulary Expansion."



%
%
% Page size stuff
%
%

\def\pgfsys@papersize#1#2{\pgf@sys@fail{papersize}}
% This command sets the paper size in a portable way. It tries to do
% so "locally" for the current page, but what really happens depends
% strongly on the driver. For instance, dvips will use whatever is
% used with the first call to this command that is part of the
% shipped-out document and will ignore all other uses. In contrast,
% pdftex will use the current value of the papersize for each page
% and, additionally, setting the papersize is local to the current TeX
% group.
% #1 = width
% #2 = height

\def\pgfsys@global@papersize#1#2{\pgfsys@papersize{#1}{#2}}
% This command does the same as the papersize command, only for
% drivers that support this, \global is used to set the papersize.
% #1 = width
% #2 = height

\def\pgfsys@prepare@papersize#1#2{\pgfsys@papersize{#1}{#2}}
% Like \pgfsys@papersize, this prepares the paper size in a portable
% way -- but it is supposed to be a preparation step before the
% document has started. Drivers may decide to postpone the action
% until the document has started.
%
% The root cause why I added this method is that image externalization
% seems to suffer from the fact that some drivers need a "prepared"
% routine whereas others must not have a prepared routine and have to
% apply the action right-away. This method is the solution for image
% externalization.
% #1 = width
% #2 = height

\ifx\paperheight\@undefined
  \def\pgfsys@thepageheight{\the\vsize} % this is wrong, but I do not
                                % know how to compute the page height
                                % in plain tex.
  \def\pgfsys@thepagewidth{\the\hsize}
\else
  \def\pgfsys@thepageheight{\paperheight}
  \def\pgfsys@thepagewidth{\paperwidth}
\fi


%
%
% Position tracking commands
%
%

% Marks the current position on the page. This position can later be
% referenced using \pgfsys@getposition
\def\pgfsys@markposition#1{\pgf@sys@fail{marking the current position}}

% Sets #2 to a pgfpoint macro that returns the position #1 is
% returned in a coordinate system where the x-axis goes right and the
% y-axis goes up. The origin of this coordinate system may or may not
% lie in the lower left corner; you should shift the coordinate system
% to the position "pgfpageorigin" to make sure that the origin is at
% the lower left corner. Typically, getting this position
% requires two runs of TeX. On the first run, the macro will be set to
% \relax, indicating that the position is not available.
%
% If #1 is set to the special position "pgfpageorigin", the position
% of the the lower left page corner is returned.

\def\pgfsys@getposition#1#2{\let#2=\relax}



%
%
% Commands having a default implementation. These commands only need
% not be implemented by drivers if this results in a better
% performance.
%
%


\def\pgfsys@shadingoutsidepgfpicture#1{%
  #1%
}
% This command gets the definition of a shading (@pgfshadingxxxx!) as
% an argument and should insert this shading into the normal text.
%
% This command has a default implementation.


\def\pgfsys@shadinginsidepgfpicture#1{%
  \pgf@makehbox{#1}%
  \setbox\pgf@hbox=\hbox{\hskip-.5\wd\pgf@hbox\lower.5\ht\pgf@hbox\hbox{\box\pgf@hbox}}%
  \pgfqbox{\pgf@hbox}%
}
% This command works like \pgfsys@shadingoutsidepgfpicture, only it is
% called when a shading should be inserted inside a picture. This
% command should insert the shading centered on the origin (which will
% typically have been low-level translated somewhere else.)
%
% This command has a default implementation.


\def\pgfsys@hbox#1{%
  \pgfsys@begin@idscope%
  \pgfsys@beginscope%
    \setbox#1=\hbox{\box#1}%
    \wd#1=0pt%
    \ht#1=0pt%
    \dp#1=0pt%
    \box#1%
  \pgfsys@endscope%
  \pgfsys@end@idscope%
}
% Called to insert a TeX hbox into a pgfpicture.

\def\pgfsys@hboxsynced#1{%
  \pgfsys@beginscope\pgflowlevelsynccm\pgfsys@hbox#1\pgfsys@endscope%
}%
% Called to insert box #1 into a pgfpicture, but with the current
% coordinate transformation matrix synced with the canvas
% transformation matrix and surrounded by a scope.
%
% In essence, this command does the same as if you first said
% \pgflowlevelsynccm and then \pgfsys@hbox#1. However, a
% ``TeX-translation'' is used for the translation part of the
% transformation cm. This will ensure that hyperlinks ``survive'' at
% least translations.

\def\pgfsys@pictureboxsynced#1{%
  {%
    \setbox0=\hbox{\pgfsys@beginpicture\box#1\pgfsys@endpicture}%
    \pgfsys@hboxsynced0%
  }%
}
% Basically, this should do the same as doing a (scoped) low level sync
% followed by inserting the box #1 directly into the output
% stream. However, the default implementation uses \pgfsys@hboxsynced
% in conjunction with \pgfsys@beginpicture to ensure that, if
% possible, hyperlinks survive in pdfs. Drivers that are sensitive to
% picture-in-picture scopes should replace this implementation by
% \pgfsys@beginscope\pgflowlevelsynccm\box#1\pgfsys@endscope

\def\pgfsys@beginpicture{}
% Called at the beginning of a pgfpicture. Used to setup things.

\def\pgfsys@endpicture{}
% Called at the end of a pgfpicture.

\def\pgfsys@typesetpicturebox#1{%
  \pgf@ya=\pgf@shift@baseline\relax%
  \advance\pgf@ya by-\pgf@picminy\relax%
  %
  %
  \advance\pgf@picmaxy by-\pgf@picminy\relax% maxy is now the height
  \advance\pgf@picmaxx by-\pgf@picminx\relax% maxx is now the width
  \setbox#1=\hbox{\hskip-\pgf@picminx\lower\pgf@picminy\box#1}%
  \ht#1=\pgf@picmaxy%
  \wd#1=\pgf@picmaxx%
  \dp#1=0pt%
  \leavevmode%
  \pgf@xa=\pgf@trimleft@final\relax  \ifdim\pgf@xa=0pt \else\kern\pgf@xa\fi
  \raise-\pgf@ya\box#1%
  \pgf@xa=\pgf@trimright@final\relax \ifdim\pgf@xa=0pt \else\kern\pgf@xa\fi
}
% Called after a picture has been typeset in box #1. This
% command should insert the box into the normal TeX code. The box #1
% will still be a ``raw'' box that contains only the \special's. The
% job of this command is to resize and shift this box according to the
% baseline shift and the size of the box.
%
% This command has a default implementation.

\def\pgfsys@beginpurepicture{\pgfsys@beginpicture}
% This version of the beginpicture command can be used for pictures
% that are guaranteed not to contain any hboxes. In this case, a
% driver might provide a more compact version of the command.

\def\pgfsys@endpurepicture{\pgfsys@endpicture}
% Called at the end of a ``pure'' pgfpicture.

\def\pgfsys@defineimage{%
  \ifx\pgf@imagewidth\pgfutil@empty\else\edef\pgf@imagewidth{width=\pgf@imagewidth,}\fi%
  \ifx\pgf@imageheight\pgfutil@empty\else\edef\pgf@imageheight{height=\pgf@imageheight,}\fi%
  \ifx\pgf@imagepage\pgfutil@empty\else\edef\pgf@imagepage{page=\pgf@imagepage,}\fi%
  \edef\pgf@image{\noexpand\includegraphics[\pgf@imageheight\pgf@imagewidth]{\pgf@filename}}%
}
% Called, when an image should be defined.
%
% This command does not take any parameters. Instead, certain macros
% will be preinstalled with appropriate values when this command is
% invoked. These are:
%
% - \pgf@filename
%   File name of the image to be defined.
%
% - \pgf@imagewidth
%   Will be set to the desired (scaled) width of the image.
%
% - \pgf@imageheight
%   Will be set to the desired (scaled) height of the image.
%
%   If this macro and also the height macro are empty, the image
%   should have its `natural' size.
%
%   If exactly only of them is specified, the undefined value the
%   image is scaled so that the aspect ratio is kept.
%
%   If both are set, the image is scaled in both directions
%   independently, possibly changing the aspect ratio.
%
%
% The following macros presumable mostly make sense for drivers that
% can handle pdf:
%
% - \pgf@imagepage
%   The desired page number to be extracted from a multi-page
%   `image'.
%
% - \pgf@imagemask
%   If set, it will be set to `/SMask x 0 R', where x is the pdf
%   object number of a soft mask to be applied to the image.
%
% - \pgf@imageinterpolate
%   If set, it will be set to `/Interpolate true' or  `/Interpolate
%   false', indicating whether the image should be interpolated in
%   pdf.
%
%
% The command should now setup the macro \pgf@image such that calling
% this macro will result in typesetting the image. Thus, \pgf@image is
% the `return value' of the command.


\def\pgfsys@defobject#1#2#3#4{%
  \pgfsysprotocol@getcurrentprotocol\pgfsys@temp%
  {%
    \pgfpicturetrue%
    \pgfsysprotocol@setcurrentprotocol\pgfutil@empty%
    \pgfsysprotocol@bufferedtrue%
    \pgfsys@beginscope%
    #4%
    \pgfsys@endscope%
    \pgfsysprotocol@getcurrentprotocol\pgfsys@@temp%
    \expandafter\global\expandafter\let\csname #1\endcsname=\pgfsys@@temp%
  }%
  \pgfsysprotocol@setcurrentprotocol\pgfsys@temp%
}
% Declares an object for later use. The idea is that the object can be
% precached in some way and then be rendered more quickly when used
% several times. For example, an arrow head might be defined and
% prerendered in this way.
%
% The parameter #1 is the name for later use. #2 and #3 are pgf points
% specifying a bounding box for the object. #4 is the code for the
% object. The code may only contain protocolable code.


\def\pgfsys@useobject#1#2{%
  \pgfsysprotocol@getcurrentprotocol\pgfsys@temp%
  {%
    \pgfsysprotocol@setcurrentprotocol\pgfutil@empty%
    \pgfsysprotocol@bufferedfalse%
    #2%
    \expandafter\pgfsysprotocol@setcurrentprotocol\csname #1\endcsname%
    \pgfsysprotocol@invokecurrentprotocol%
  }%
  \pgfsysprotocol@setcurrentprotocol\pgfsys@temp%
}
% Renders a previously declared object. The first parameter is the
% name of the object. The second parameter is extra code that
% should be executed right *before* the object is rendered. Typically,
% this will be some transformation code.


\def\pgfsys@begininvisible{\pgfsys@transformcm{1}{0}{0}{1}{2000bp}{2000bp}}
% Between this command and the closing endinvisible, all output should
% be suppressed. Nothing should be drawn at all, which includes all
% paths, images and shadings. This command typically should not open a
% graphic scope.

\def\pgfsys@endinvisible{\pgfsys@transformcm{1}{0}{0}{1}{-2000bp}{-2000bp}}
% Ends the invisibility section, unless invisibility blocks have been
% nested. In this case, only the `last' one restores visibility.

\def\pgfsys@begininvisiblescope{\pgfsys@beginscope\pgfsys@begininvisible}
% Works like pgfsys@begininvisible, but it may open a graphics scope.

\def\pgfsys@endinvisiblescope{\pgfsys@endinvisible\pgfsys@endscope}
% Closes the invisibility scope.


\def\pgfsys@atbegindocument{}
% This command will be executed at the begin of the document. If for
% whatever reason this is not done, you may wish to execute this
% command yourself. It will setup certain specials that need to be
% inserted into the resulting .pdf/.ps/.xxx file.


% String comparison
\ifdefined\pdfstrcmp
  \let\pgfsys@strcmp\pdfstrcmp
\else\ifdefined\strcmp
  \let\pgfsys@strcmp\strcmp
\else\ifdefined\directlua
  \directlua{
  local lft = lua.get_functions_table()
  lft[\string#lft+1] = function()
      local lhs = token.scan_string()
      local rhs = token.scan_string()
      if lhs < rhs then
          tex.sprint(-2, "-1")
      elseif lhs == rhs then
          tex.sprint(-2, "0")
      else
          tex.sprint(-2, "1")
      end
  end
  token.set_lua("pgfsys@strcmp", \string#lft, "global")
  }
\else
  \def\pgfsys@strcmp#1#2{\pgf@sys@fail{string comparison}}%
\fi\fi\fi


% Discern the driver:

% Default driver:
\ifx\pgfsysdriver\@undefined
  \pgfutil@guessdriver
\fi

\input pgf.cfg

\immediate\write-1{Driver file for pgf: \pgfsysdriver}
\pgfutil@InputIfFileExists{\pgfsysdriver}{}{%
  \pgferror{Driver file ``\pgfsysdriver'' not found.}}



% This will insert always-present-specials at the beginning of the
% code.
\AtBeginDocument{
  \pgfsys@atbegindocument
  \let\pgfsys@atbegindocument=\pgfutil@empty
}



% Read aux file in plain and context mode:

\pgfutil@aux@read@hook


\endinput

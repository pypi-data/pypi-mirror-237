\documentclass[fontsize=12pt,parskip=half]
              {scrartcl}

\usepackage[ngerman]{babel}

\usepackage[T1]{fontenc}
% Hinweis: Nur einer der beiden folgenden Zeilen wird benötigt.
\usepackage{lmodern}
\usepackage{charter,helvet}

\usepackage{enumerate}

\usepackage[clausemark=forceboth,
            juratotoc, juratocnumberwidth=2.5em]
           {scrjura}
\useshorthands{'}
\defineshorthand{'S}{\Sentence\ignorespaces}
\defineshorthand{'.}{. \Sentence\ignorespaces}

\pagestyle{myheadings}

\begin{document}

\subject{Satzung}
\title{VfVmai}
\subtitle{Verein für Vereinsmaierei mit ai n.\,e.\,V.}
\date{11.\,11.\,2011}
\maketitle

\tableofcontents

\addsec{Präambel}

Die Vereinslandschaft in Deutschland ist vielfältig.
Doch leider mussten wir feststellen, dass es dabei oft
am ernsthaften Umgang mit der Ernsthaftigkeit krankt.

\appendix

\section{Allgemeines}

\begin{contract}
\Clause{title={Name, Rechtsform, Sitz des Vereins}}

Der Verein führt den Namen »Verein für Vereinsmaierei mit 
ai n.e.V.« und ist in keinem Vereinsregister eingetragen.

'S Der Verein ist ein nichtwirtschaftlicher, unnützer
Verein'. Er hat keinen Sitz und muss daher stehen.

Geschäftsjahr ist vom 31.~März bis zum 1.~April.

\Clause{title={Zweck des Vereins}}

'S Der Verein ist zwar sinnlos, aber nicht zwecklos'.
Vielmehr soll er den ernsthaften Umgang mit der
Ernsthaftigkeit auf eine gesunde Basis stellen.

Zu diesem Zweck kann der Verein
\begin{enumerate}[\qquad a)]
\item in der Nase bohren,
\item Nüsse knacken,
\item am Daumen lutschen.
\end{enumerate}

Der Verein ist selbstsüchtig und steht dazu.

Der Verein verfügt über keinerlei Mittel.\label{a:mittel}

\Clause{title={Vereinsämter}}

Die Vereinsämter sind Ehrenämter.

'S Würde der Verein über Mittel verfügen 
(siehe \ref{a:mittel}), so könnte er einen
hauptamtlichen Geschäftsführer bestellen'. Ohne
die notwendigen Mittel ist dies nicht möglich.

\Clause{title={Vereinsmaier},dummy}
\label{p.maier}
\end{contract}

\section{Mitgliedschaft}

\begin{contract}
\Clause{title={Mitgliedsarten},dummy}

\Clause{title={Erwerb der Mitgliedschaft}}

Die Mitgliedschaft kann jeder zu einem angemessenen 
Preis von einem der in \refClause{p.maier}
genannten Vereinsmaier erwerben.\label{a.preis}

'S Zum Erwerb der Mitgliedschaft ist ein formloser
Antrag erforderlich'. Dieser Antrag ist in grüner
Tinte auf rosa Papier einzureichen.

Die Mitgliedschaft kann nicht abgelehnt werden.

\SubClause{title={Ergänzung zu vorstehendem 
    Paragraphen}}

'S Mit Abschaffung von \refClause{p.maier} verliert
\ref{a.preis} seine Umsetzbarkeit'. Mitgliedschaften
können ersatzweise vererbt werden.

\Clause{title={Ende der Mitgliedschaft}}

'S Die Mitgliedschaft endet mit dem Leben'. Bei nicht
lebenden Mitgliedern endet die Mitgliedschaft nicht.

\Clause{title={Mitgliederversammlung}}

Zweimal jährlich findet eine Mitgliederversammlung statt.

Der Abstand zwischen zwei Mitgliederversammlungen 
beträgt höchstens 6~Monate, 1~Woche und 2~Tage.

Frühestens 6~Monate nach der letzten Mitgliederversammlung
hat die Einladung zur nächsten Mitgliederversammlung zu 
erfolgen.

\SubClause{title={Ergänzung zur Mitgliederversammlung}}

Die Mitgliederversammlung darf frühstens 2~Wochen nach
letztem Eingang der Einladung abgehalten werden.
\end{contract}

\section{Gültigkeit}

\begin{contract}
\Clause{title={In Kraft treten}}

Diese Satzung tritt am 11.\,11.\,2011 um 11:11~Uhr 
in Kraft.

'S Sollten irgendwelche Bestimmungen dieser Satzung im
Widerspruch zueinander stehen, tritt die Satzung am
11.\,11.\,2011 um 11:11~Uhr und 11~Sekunden wieder 
außer Kraft'. Der Verein ist in diesem Fall als 
aufgelöst zu betrachten.

\end{contract}

\end{document}
